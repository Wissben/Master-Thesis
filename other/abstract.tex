\begin{center}
    \Large 
    \textbf{Résumé}
\end{center}
\setlength{\parindent}{0.4cm}
Les récents progrès de l'apprentissage automatique montrent clairement que l'intelligence artificielle dominera l'avenir.
L'interaction homme-machine a particulièrement attiré l'attention des plus grandes industries informatiques du monde. Ceci est particulièrement vrai avec la poussée de l'apprentissage profond qui a permis aux ordinateurs d'atteindre un niveau élevé de compréhension de la langue.
\par 
Néanmoins, ces grandes avancées n'ont pas encore atteint le niveau de communication humain et semblent être très en retard.
Les ordinateurs sont soit limités à la gestion de tâches spécifiques, telles que des assistants intelligents, soit dépourvus de compréhension linguistique réelle, tels que les chat-bots.
\par 
À travers ce travail, nous nous familiarisons avec les systèmes de gestion de dialogue vocal et leurs composants.
En tirant partie des approches d'intelligence artificielle, allant des modèles acoustiques, à la génération de langage naturel, tout  en passant par la compréhension du langage naturel et la  gestion de dialogue; notre solution peut aider l'utilisateur à manipuler l'ordinateur à l'aide de directives vocales.
\par
\noindent
\textbf{Mots clés :} Assistant personnel intelligent, Apprentissage par renforcement, Apprentissage profond, Intelligence artificielle, Interaction homme-machine, Compréhension du langage naturel, Classification d'intention, Reconnaissance automatique de la parole, Gestion de dialogue, Bases de connaissances.

\vspace*{0.8cm}

\begin{center}
    \Large 
    \begin{arab}
    ملخص
    \end{arab}
\end{center}
\begin{arab}
	أوضحت التطورات الحديثة في مجال التعلم الآلي أنّ الذكاء الاصطناعي سيهيمن على حياة الإنسان في المستقبل القريب. وعلى وجه الخصوص، ألقى عمالقة الإعلام الآلي تركيزًا خاصًا على مجال التفاعل بين الإنسان والآلة. فقد حصل ذلك تزامنًا مع تطور التعلم العميق الذي مهد الطريق أمام أجهزة الكمبيوتر للوصول إلى مستوى عالٍ من فهم اللغة.
	
	ومع ذلك ، فإن هذه التطورات العظيمة لم تصل بعد إلى المستوى المنشود و تبقى بعيدة عن طرق تواصل البشر الطبيعية فيما بينهم إذ تقتصر أجهزة الكمبيوتر إما على إدارة مهام محددة مثل المساعدين الأذكياء أو تفتقر إلى الفهم الفعلي للّغة.
	
	سنتعرف من خلال هذا العمل على أنظمة الحوار ونفهم مكوناتها الأساسية. لنستطيع بذلك و باستخدام أساليب الذكاء الاصطناعي، من النماذج الصوتية إلى توليد النص مرورًا بفهم اللغة و إدارة الحوار ، أن نقدم حلاً  يمَّكِن المستخدم من التحكم في الكمبيوتر عن طريق توجيهات شفوية.
\par 
\textbf{الكلمات الدالة :} مساعد شخصي ذكي ، التعلم العميق ، الذكاء الاصطناعي ، التفاعل بين الإنسان والحاسوب ، فهم اللغة الطبيعية ، تصنيف النوايا ، التعرف التلقائي على الكلام ، إدارة الحوار ، قواعد المعرفة.
\end{arab}

\vspace*{0.8cm}

\newpage
\clearpage
\begin{center}
    \Large 
    \textbf{Abstract}
\end{center}

\paragraph{}
Recent advances in machine learning are making it clear that the near future will be dominated by artificial intelligence. In particular, human-machine interaction received a great focus from the world's biggest computer science industries. This is especially true with the surge of deep learning which paved the way for computers to reach a high level of language understanding.

Nonetheless, these great advances did not yet reach human level communications and seem to be lagging behind. Computers are either restricted to the management of specific tasks like smart assistants or lack the actual language understanding like chatbots.

Through this work, We familiarize ourselves with spoken dialogue systems and understand their components. By taking advantage of artificial intelligence approaches, ranging from acoustic models, to natural language generation, while going through natural language understanding and dialogue management; our solution can assist the user with computer manipulation using spoken directives.

\noindent
\textbf{Keywords :} Smart Personal Assistant, Deep Learning, Reinforcement Learning, Artificial Intelligence, Human-Computer Interaction, Natural Language Understanding, Intent Classification, Automatic Speech Recognition, Dialog Management, Knowledge Bases.