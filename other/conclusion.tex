\chapter*{Conclusion générale}

%\paragraph{}
%La création d'un nouveau besoin qu'est la personnalisation des services électroniques et numériques, a été la principale source de motivation pour ce projet. Cependant, et par faute de temps ainsi que de moyens techniques, surtout en ce qui concerne le côté matériel pour l'apprentissage automatique, certains modules n'ont pas été perfectionnés. Cela ne nous a pas empêché de réaliser un travail dont nous sommes particulièrement fières. Mais, nous gardons toutefois un esprit critique, ainsi qu'une objectivité envers le travail fourni. 

\paragraph{}
Tout au long de la réalisation de ce mémoire, nous avons étudié l'état actuel des assistants personnels intelligents. Nous avons dû passer une majeure partie de cette étape à comprendre les fondements théoriques et conceptuels de chaque composant de ces systèmes, principalement à cause de la grande densité de techniques, concepts et théories qui sont nouvelles pour nous. Cela ne nous a pas empêchés de réaliser un travail dont nous sommes particulièrement fiers. Mais, nous gardons toutefois un esprit critique ainsi qu'une objectivité envers le travail fourni. 

\paragraph{}
Après avoir assimilé la totalité des concepts, et qui font office d'état de l'art du domaine, nous sommes arrivés à certaines conclusions. Tout d'abord, développer un système en partant de rien était un travail assez massif, dépassant de loin le cadre d'un projet de fin d'études de master. Quatre modules ont été développés : le module de reconnaissance automatique de la parole, le module de compréhension automatique du langage naturel, le module de gestion du dialogue et le module de génération du langage naturel. L'accent à été mis sur certains d'entre eux comme le module de compréhension du langage naturel et le module de gestion du dialogue. Ces derniers dépendaient principalement de notre problématique. Le module de reconnaissance automatique de la parole a été sujet à une amélioration spécifique à nos besoins tout en exploitant un système de base déjà existant, à savoir DeepSpeech.


\paragraph{}
Avec une idée claire du travail à réaliser, nous avons pu entamer la conception de chaque module en y incorporant nos ajouts et modifications. Beaucoup de ces modifications sont le fruit de longues séances de débats et de discussions. 

%\paragraph{}
%L'étape de réalisation est celle qui a pris le plus de temps. Entre la découverte de nouvelles technologies à utiliser, la collecte des données et l'apprentissage des différents modèles.
\paragraph{}
En ce qui concerne le module de reconnaissance automatique de la parole, l'ajout de notre modèle de langue a amélioré les résultats. Cela s'accordait avec nos prédictions théoriques. La partie de construction du corpus pour ce modèle a été soumise à beaucoup d'optimisations incrémentales, en tombant à chaque fois sur un nouveau problème, ou bien un obstacle matériel (manque de puissance de calcul). Par conséquent, les résultats n'étaient pas assez encourageants, surtout si le but est de concurrencer les systèmes propriétaires comme celui de Google qui réalise un score quasi-parfait sans apprentissage supplémentaire.
%Perspective ASR
%------------------------

%----------------------
\paragraph{}


Pour le module de compréhension automatique du langage naturel, les ajouts faits au modèle d'apprentissage ont été expérimentalement validés dans le chapitre "Réalisations et résultats". L'ajout de l'information morphosyntaxique aux vecteurs d'encodage de chaque mot de la requête a enrichi la codification de cette dernière. L'introduction d'erreurs aléatoires dans le corpus d'apprentissage a quand à elle permis la gestion d'éventuelles erreurs que pourrait engendrer le module de reconnaissance automatique de la parole. La construction de l'ensemble d'apprentissage à partir de zéro était l'étape la plus longue de la réalisation de ce module. Plus l'ensemble grandissait, plus il était difficile de maintenir sa validité sans l'intervention d'un soutien externe. De plus, vu que la tâche à accomplir était relativement simple et limitée, le réel impact de cet ajout ne peut pas être certifié et validé dans un cadre plus général. Une autre problématique est celle du manque de données d'apprentissage consacrées au domaine de la manipulation d'ordinateurs. Ces données sont généralement construites manuellement par les développeurs du système. Un autre point à soulever est celui du manque de diversité dans les tâches réalisables par l'assistant. Cet ensemble de tâches reste facilement extensible. Il suffit d'ajouter des exemples assez exhaustifs à l'ensemble de données. Cela reste néanmoins une tâche lourde et manuelle à plus grande échelle.

%NLU perspective
%---------------------

%----------------------
\paragraph{}
En ce qui concerne le gestionnaire de dialogue, nous l'avons conçu pour qu'il soit facilement ajustable et mis à l'échelle. L'utilisation d'une architecture hiérarchisée d'agents de dialogue permet de réduire grandement la complexité de développement et d'ajout d'un nouveau gestionnaire de tâches dans le système.  Pour représenter l'état interne d'une telle architecture, nous avons utilisé des graphes de connaissances au lieu des trames sémantiques dont la flexibilité permet de représenter tout l'état du dialogue des différents agents de l'architecture simultanément. Pour ce qui est des agents de dialogue, ils sont entraînés en utilisant des techniques d'apprentissage par renforcement. Ils interagissent avec un simulateur d'utilisateur afin d'atteindre un but à travers l'optimisation d'une politique d'actions basée sur un système de récompenses.

Cependant, l'utilisation du graphe de connaissances et de l'apprentissage profond a un prix. En effet, la codification de la totalité du graphe en un seul vecteur de taille fixe était une tâche assez difficile. La taille nécessaire de ce dernier devrait augmenter exponentiellement avec l'ajout de nouvelles connaissances. De plus, la codification choisie pour les n\oe{}uds du graphe a eu pour effet de faire perdre de l'information sémantique. Le décodage du graphe s'en est trouvé grandement affecté.

%----------

%---------

%\paragraph{}
%Pour le dernier module, à savoir le module de génération du langage naturel. Il %fût assez simple à réaliser. Son fonctionnement est très rudimentaire et %intuitif. Il s'agissait d'utiliser un ensemble de phrases modèles et de %remplacer les valeurs manquantes avec des valeurs réelles.


\paragraph{}
Pour ce qui est de l'application, nous avons fait le choix d'utiliser une architecture trois tiers basée web. Ce choix fût motivé par le fait que l'utilisation d'un serveur offre une puissance de calcul considérablement plus élevée que celle d'une machine personnelle en local. L'interface reste assez simple et épurée. Le but est de prioriser la parole comme moyen de communication avec le système. Cependant, cette interface a aussi pour but de montrer les fonctionnalités du système. Elle est donc plus orientée vers le développement.



\paragraph{}
En ce qui concerne les perspectives envisageables pour ce travail, nous avons longuement réfléchis à des alternatives possibles pour certains modules.
\par
Premièrement, l'avenir de systèmes Open Source pour la reconnaissance automatique de la parole est très prometteur. Ces derniers offrent un moyen libre de mener des études et contribuer au développement à grande échelle de cette discipline. Une perspective future pour ce module serait de lancer notre propre plateforme de collecte de données pour le modèle acoustique et le modèle de langue. Cette idée a déjà été discutée dans la partie de l'état de l'art. Malheureusement, le temps a cruellement manqué. L'investissement de la communauté dans de telles initiatives n'en reste pas moins indéniable, comme l'a démontré le projet CommonVoice de Mozilla.
\par
Deuxièmement, pour le module de compréhension automatique du langage naturel, faire appel à une collecte massive de données est une solution explorable. Le développement d'un outil d'aide à l'annotation d'un corpus était aussi le sujet d'un long débat. Le manque de temps nous a poussé à retarder son développement. La mise à l'échelle d'une telle plateforme pourrait grandement faire avancer la tâche fastidieuse qu'est la collecte de données. 
%L'automatisation de l'ajout de nouvelles fonctionnalités, comme de nouvelles intentions ou nouvelles étiquettes, n'a pas été discutée pendant la réalisation de ce travail. Mais, une intégration à la plateforme de collecte de données précédemment citée serait une solution de départ.
\par 

Ensuite, en ce qui concerne le module de gestion du dialogue, il est envisageable d'utiliser une méthode d'apprentissage semi-supervisée pour l'encodage des n\oe{}uds du graphe. Cet ajout pourrait permettre d'enrichir la valeur sémantique de ces n\oe{}uds, et facilitera la tâche au réseau de neurones de l'agent apprenant pour le décodage du graphe, principalement grâce grâce au fait que les n\oe{}uds dont les sens sont arbitrairement proches auront des codifications similaires.

\par
En ce qui concerne le module de génération du langage naturel, une méthode plus sophistiquée comme l'utilisation de modèles d'apprentissage automatique basés encodeur-décodeur, ou bien un convertisseur de graphes de connaissances en texte pourraient être utilisés. Cependant, ces architectures requièrent un très grand volume de données d'apprentissage annotées et spécifiques à notre problématique.

\par
Enfin, pour ce qui est de l'application principale, une amélioration possible serait le déploiement du serveur dans un service de Cloud Hosting. Cela permettrait de minimiser les temps d'inférence et de post-traitement. Le développement d'une interface plus légère et plus orientée vers les cas pratiques est une amélioration possible. Garder les deux cas de figures, c.à.d. utilisation et développement, est aussi possible.

\paragraph{}
Pour conclure, nous estimons que la totalité du projet était une énorme occasion d'approfondir nos connaissances, que ce soit celles qui nous ont été enseignées durant notre cursus, comme l'apprentissage automatique, le traitement automatique du langage naturel, le web-sémantique et la représentation de connaissances, ou bien celles que nous avons apprises au cours de notre étude de la littérature, comme l'apprentissage par renforcement, le traitement automatique de la parole, le nettoyage des données, etc. Ce projet nous a aussi initiés au travail en équipe pour la réalisation d'un projet assez conséquent, tout en étant encadrés par nos supérieurs.
Finalement, nous pensons que la plus grande satisfaction vient du fait que nous avons réalisé, dans un certain délai restreint, un travail qui traite d'un sujet récent et ambitieux, et ainsi, de poser notre propre pierre à l'édifice pour, idéalement, encourager les chercheurs en Algérie à s'investir dans ce domaine. 

