\chapter*{Introduction générale}
\paragraph{}
\begin{chapquote}{Amy Stapleton, Opus Research 2015}
	\say{Nous entrons dans un nouveau monde. Les technologies d'apprentissage automatique, de reconnaissance de la parole et de compréhension du langage naturel atteignent un niveau de capacités supérieur. Le résultat final est que nous aurons bientôt des assistants intelligents pour nous aider dans tous les aspects de nos vies.}
\end{chapquote}

\paragraph{}
Plus de quarante ans se sont écoulés depuis la présentation du premier assistant virtuel commandé vocalement par la compagnie IBM \citep{ibm_spa}. Déjà à cette époque là, ce fut présenté comme une révolution technologique. Un programme qui pouvait reconnaître 16 mots et les chiffres de 0 à 9. Quelques décennies plus tard, nous nous retrouvons avec des assistants capables de reconnaître, comprendre et parler plusieurs langues. Ces assistants intelligents sont la nouvelle génération d'intelligence artificielle capable de s'intégrer dans nos vies personnelles et professionnelles. Les assistants virtuels intelligents (Smart Personal Assistants, SPA) sont un exemple de cette	 intelligence artificielle. 

\paragraph{}
Un domaine d'application pour les SPAs qui a particulièrement émergé est celui de l'aide à la manipulation d'un ordinateur personnel. Selon certains experts \citep{spa_arch,virtualbutler,SPA-overview}, l'époque où nous utilisions encore le clavier, la souris et l'écran est une étape de transition. Le futur se trouve dans l'exploitation de la parole comme moyen de communication principal avec les machines; et ce futur est proche. La course à la réalisation d'assistants qui excellent dans plusieurs domaines a commencé il y a quelques années avec l'entrée de grandes compagnies comme Google et Apple dans le secteur \citep{spas_survey}. Par la suite, de très grands efforts ont été fournis dans le but d'améliorer l'expérience d'utilisation de ces assistants. Les chercheurs ainsi que les industriels se sont orientés vers cette solution très rapidement, notamment en investissant dans la recherche de solutions exploitant l'apprentissage automatique \citep{spas_survey}.

\paragraph{}
Nous sommes sans doute actuellement entrain de vivre une époque importante de l'intelligence artificielle, avec le développement de l'apprentissage automatique, surtout dans le domaine de la reconnaissance automatique de la parole, de la compréhension automatique du langage naturel et plus récemment de l'apprentissage par renforcement \citep{SPA-overview}. L'accomplissement de notre rêve de converser avec une machine semble à la fois proche en termes de temps, mais loin en termes de progrès. Les utilisateurs réguliers de produits technologiques sont confrontés à une technologie nouvelle mais prometteuse. Ce secteur d'activités peut donc s'avérer très prolifique si les efforts fournis sont suffisamment conséquents.

\paragraph{}
C'est dans ce contexte que s'inscrivent les travaux de notre projet de fin d'études de master. En effet, pour ne pas être en marge de l'évolution dans ce secteur qu'est la personnalisation des services électroniques, cette thématique avec l'application de techniques d'apprentissage automatique, qui sont la base de nombreux systèmes actuels, Google Assistant, Apple Siri,Microsoft Cortana \citep{spas_battle_royal}, et des systèmes Open source comme Rasa \citep{rasa_nlu}, devrait être un défi  pour l'Algérie en général et  pour nous en particulier. Très motivés par l'envie de nous initier à ce domaine, nous espérons ainsi apporter notre modeste contribution. Nous pensons aussi que les plus ambitieux des projets commencent avec des contributions à petites échelles. Notre assistant aura pour but d'améliorer l'expérience d'utilisation d'un ordinateur, ceci en effectuant des tâches rudimentaires, efficacement et sans réel effort de la part de l'utilisateur hormis l'énonciation de la requête. En utilisant des techniques d'intelligence artificielle d'actualité comme la reconnaissance automatique de la parole, la compréhension du langage naturel et l'apprentissage par renforcement, ce projet se veut assez ambitieux et vise à faire gagner du temps à tout utilisateur d'un ordinateur de bureau ou portable.
	
\paragraph{}
Dans cette optique, nous nous sommes inspirés dans notre étude des travaux de la littérature sur les assistants personnels intelligents. Nous passerons en revue les aspects théoriques qui sont utilisés dans des solutions considérées comme état de l'art du domaine. Nous nous baserons sur ces techniques pour la conception des modules de notre système tout en les adaptant à nos besoins. 

\paragraph{}
Ce mémoire se constitue de quatre chapitres. Le premier chapitre sera consacré à la présentation des assistants virtuels intelligents. Le deuxième chapitre se focalisera sur l'étude des travaux existants liés à la thématique de notre sujet comme l'utilisation de l'apprentissage profond, l'apprentissage par renforcement et les ontologies pour le développement d'un assistant virtuel intelligent. Le troisième chapitre traitera de l'étude conceptuelle. Le quatrième et dernier chapitre présentera notre système Speact avec son évaluation, une synthèse des résultats obtenus, et notre interface pour l'application. Enfin, nous clôturerons ce travail avec une conclusion générale et les perspectives envisageables.
