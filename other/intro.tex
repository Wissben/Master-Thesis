\chapter*{Introduction générale}
\paragraph{}
\chapterprecishere{"Nous entrons dans un nouveau monde. Les technologies d'apprentissage automatique, de reconnaissance de la parole et de compréhension du langage naturel atteignent un niveau de capacités. Le résultat final est que nous aurons bientôt des assistants intelligents pour nous aider dans tous les aspects de nos vies."\par\raggedleft--- \textup{Amy Stapleton,}, Opus Research}

\paragraph{}
Plus de quarante ans se sont écoulés depuis la présentation du premier assistant virtuel commandé vocalement par la compagnie IBM. Déjà à cette époque là, ce fut présenté comme une révolution technologique. Un programme qui pouvait reconnaître 16 mots et les chiffres de 0 à 9. Quelques générations plus tard, nous nous retrouvons avec des assistants capables de reconnaître, comprendre et parler plusieurs langues. Ces assistants intelligents sont la nouvelle génération d'intelligences artificielles capables de s'intégrer dans nos vies personnelles et professionnelles. 

\paragraph{}
Un domaine qui à particulièrement émergé est celui de l'aide à la manipulation d'un ordinateur personnel. Selon certains experts, l'époque où nous utilisons encore le clavier, la souris et l'écran est une étape de transition. Le futur se trouve dans l'exploitation de la parole comme moyen de communication principal avec les machines. Et ce futur est proche, la course à la réalisation d'assistants qui excellents dans plusieurs domaines a commencé il y a quelques années avec l'entrée de grandes compagnies comme Google et Apple dans le secteur. Par la suite, de très grands efforts ont étés fournis dans le but d'améliorer l'expérience d'utilisation de ces assistants. Les chercheurs ainsi que les industriels se sont orientés vers cette solution très rapidement.

\paragraph{}
En tant que novice dans ce secteur qu'est la personnalisation des services électroniques, l'Algérie devra rapidement se positionner pour s'incorporer dans l'évolution de ces technologies. Nous avons donc étés motivés par l'envie de nous initier à ce domaine, ainsi que de motiver d'autres compatriotes scientifiques à continuer sur cette voie. Nous pensons que les plus ambitieux des projets commencent avec des contributions à petites échelles. Notre assistant aura pour but d'améliorer l'expérience d'utilisation d'un ordinateur. Ceci en effectuant des tâches rudimentaires, efficacement et sans réel effort hormis l'énonciation de la requête. En utilisant des techniques d'intelligence artificielle d'actualité comme la reconnaissance automatique de la parole, la compréhension du langage naturel et l'apprentissage par renforcement, ce projet se veut assez ambitieux et vise à faire gagner du temps à tout utilisateur d'un ordinateur de bureau ou portable.
	
\paragraph{}
Dans cette optique, nous présenterons notre étude de la littérature concernant les assistants personnels intelligents. Nous passerons en revu les aspects théoriques qui sont utilisés dans des solutions considérées comme état de l'art du domaine. Nous nous baserons sur ces techniques pour la conception des modules de notre système tout en les adaptant à nos besoins. 

\paragraph{}
Ce mémoire se constitue de quatre chapitres. Le premier chapitre sera consacré à la présentation des assistants virtuels intelligents. Le deuxième chapitre se focalisera sur l'étude des travaux existants traitant de notre sujet. Le troisième chapitre entamera l'étude conceptuelle. Le quatrième et dernier chapitre présentera les évaluations du système réalisées, suivis d'une discussion des résultats obtenus. Enfin nous présenterons notre interface pour l'application et nous conclurons ce travail avec une conclusion générale.

