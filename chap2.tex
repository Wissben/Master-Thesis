\chapter{Composants de base d'un assistant personnel}

\section{Introduction}
\section{Architecture d'un SPA}
\section{Reconnaissance automatique de la parole (ASR)}
	\paragraph{}
	\subsection{Définition}
		\paragraph{}
	\subsection{Modèle acoustique}
		\paragraph{}
	\subsection{Modèle de langage}
		\paragraph{}
	\subsection{Méthodes utilisées}
		\paragraph{}
		\subsubsection{N-grammes}
			\paragraph{}
		\subsubsection{Modèle de Markov Caché}
			\paragraph{}
\section{Compréhension du langage naturel (NLU)}
	\paragraph{}
	\subsection{Définition}
		\paragraph{}
	\subsection{Classification d'Intent}
		\paragraph{}
	\subsection{Extraction d'entités}
		\paragraph{}
	\subsection{Analyse sémantique}

\section{Gestion du dialogue}
	\paragraph{}
	\subsection{Définition}
		\paragraph{}
	\subsection{Agent de dialogue}
		\paragraph{}
	\subsection{Apprentissage par renforcement}
		\paragraph{}
	\subsection{Simulateur d’utilisateur}
		\paragraph{}

\section{Génération du langage naturel (NLG)}
	\paragraph{}
	Le domaine de la génération automatique du langage naturel est l’un des domaines dont les bordures sont difficiles à définir (Evans et al., 2002). il est vrai que la sortie d’un tel système est clairement du texte. Cependant, l’ambiguïté se trouve dans ses l’entrées, c’est à dire, sur quoi se basera le système pour générer le texte. D’après (Reiter \& Dale, 1997)\cite{Reiter:1997} la génération du langage naturel est décrite comme étant le sous domaine de l’intelligence artificielle qui traite la construction des systèmes de génération de texte à partir d’une représentation non-linguistique de l’information, celle-ci peut être une représentation sémantique, des données numériques, une base de connaissances ou même des données visuelles (images ou vidéos). Ceci dit, d’autres travaux ,comme Labbé \& Portet (2012)\cite{Labbé2012}, utilisent les même techniques pour des entrées linguistique. Enfin la génération du langage naturel peut être très proche de la gestion de dialogue\cite{Dethlefs2014}, en effet, le texte généré doit prendre en compte l’historique de la conversation et le contexte de l’utilisateur.\newline
	Il existe six tâches trouvées fréquemment dans les systèmes de génération de texte \cite{Reiter:1997}.
	
	\subsection{Détermination du contenu}
	\paragraph{}
	Cette partie consiste à sélectionner les informations de l’entrée dont le système veut transmettre le contenu sous forme de texte naturel à l’utilisateur. En effet, les données en entrée peuvent contenir plus d’informations que ce que l’on désire communiquer\cite{Yu:2007}, de plus, cette information peut aussi dépendre de l’utilisateur et de ses connaissances\cite{Dethlefs2014}. Ce qui requiert de mettre au point un système qui détecte les informations pertinentes à l’utilisateur.
	\subsection{Structuration de texte}
	\paragraph{}
	Après la détermination du contenu, le système doit ordonner les information à transmettre. Ceci dépend grandement du domaine d’application qui peut exiger des contraintes d’ordre temporelle ou de préférence par importance des idées. Les informations à transmettre en elles-mêmes sont souvent reliées par sens ce qui implique une certaine structuration de texte à respecter.
	\subsection{Agrégation de phrases}
	\paragraph{}
	Certaines informations peuvent être transmises dans une même phrase. Cette partie introduit des notions de la linguistique afin que le texte généré soit plus lisible et éviter les répétition. Un exemple de cela peut être la description de la météo à Alger au cours de la matinée:
	\begin{itemize}
		\item Il va faire 16° à Alger à 7h.
		\item Il va faire 17° à Alger à 8h.
		\item Il va faire 18° à Alger à 9h.
		\item Il va faire 18° à Alger à 10h.
	\end{itemize}
	Ceci peut être agrégé en un texte plus compacte: "La température moyenne à Alger sera de 17° entre 7h et 10h."
	
	\subsection{Lexicalisation}
	\paragraph{}Le système choisit les mots et les expressions à utiliser pour communiquer le contenu des phrases sélectionner. La difficulté de cette tâche revient à l’existence de plusieurs manière d’exprimer la même idée. Cependant, certains mots ou expressions sont plus appropriés en certaines situations que d’autres. En effet, “inscrire un but” est une façon inadéquate d’exprimer un but contre son camp\cite{Gatt2018}.
	
	\subsection{Génération d’expressions référentielles (REG)}
	\paragraph{}Cette partie du système se focalise dans la génération d’expressions référentielles qui peuvent être entre autres: noms propres, groupes nominaux ou pronoms et ceci a pour but d’identifier les entités du domaine. Cette tâche semble être très proche de sa prédécesseur; elle s’avère néanmoins plus délicate dû à la difficulté de confier suffisamment d’information sur l’entité afin de la différencier des autres\cite{Reiter:1997}. Le système doit faire un choix de l’expression référentielle en se basant sur plusieurs facteurs, par exemple “Mohammed”, “Le professeur” ou “Il” font référence à la même personne. Cependant, le choix entre eux dépendrait de si l’entité a été mentionnée auparavant et des détails l’accompagnant par exemple. 
	
	\subsection{Réalisation linguistique}
	\paragraph{}
	La dernière tâche consiste à combiner les mots et expressions sélectionnés pour construire une phrase linguistiquement correcte. Ceci requiert l’utilisation des bonnes formes morphologiques des mots, les ordonner et éventuellement  l’addition de certains mots du langage afin de réaliser une structure de phrase grammaticalement et sémantiquement correcte. Plusieurs méthodes ont été proposées, principalement les méthodes basées sur des règles manuellement construites (modèles de phrases, systèmes basés grammaires) ou des approches statistiques \cite{Gatt2018}.
	\paragraph{Modèles de phrases:} La réalisation se fait en utilisant des modèles de phrases prédéfinis. Il suffit de remplacer des espaces réservés par certaines entrées du système. Par exemple, une application dans un contexte météorologique pourrait utiliser le modèle suivant: “la température à [ville] atteint [température]° le [date]”.\\
Cette méthode est utilisée lorsque les variations des sorties de l’application sont minimales. Son utilisation a l’avantage et l’inconvénient d’être rigide. D’un coté il est facile de contrôler la qualité des sorties syntaxiquement et sémantiquement tout en utilisant des règles de remplissage complexe \cite{Theune2001}. Cependant, lorsque le domaine d’application présente beaucoup d’incertitude, cette méthode exige un travail manuel énorme, voire impossible à faire, pour réaliser une tâche pareille. Bien que certains travaux ont essayer de faire un apprentissage de modèles de phrases à partir d’un corpus\cite{Angeli2012} cette méthode reste inefficace lorsqu’il s’agit d’application qui nécessite un grand nombre de variations linguistiques.
	\paragraph{Systèmes basés grammaire:} La réalisation peut se faire en suivant une grammaire du langage. Celle-si contient les règles morphologiques et de structures de la langues, notamment la grammaire systémique fonctionnelle (SFG)\cite{Halliday2004} a été largement utilisé comme dans NIGEL\cite{Mann1983} ou KPML\cite{Bateman1997}. L’exploitation des grammaires dans la génération du texte nécessite généralement des entrées détaillées. En plus des composantes du lexique sélectionnées, des descriptions de leurs rôles ainsi que leurs fonctions grammaticales sont souvent exigées. Un exemple d’entrée d’un système basé grammaire est celui de SURGE\cite{Elhadad1996}:
	\begin{center}
		\begin{forest} [
			[cat:clause]
			[process
			[type[composite]]
			[relation[possessive]]
			[lex[\color{red}"hand"]]
			]
			[partic
			[agent
			[cat[pers\_pro]]
			[gender[feminine]]
			]
			[affected
			[(1)
			[cat[NP]]
			[lex[\color{red}"editor"]]
			]
			]
			[possessor[(1)]]
			[possessed
			[cat[NP]]
			[lex[\color{red}"draft"]]
			]
			]
			]
		\end{forest}
	\end{center}
	Qui génère la phrase: “She hands the draft to the editor”.\newline

Comme les modèles de phrases, les systèmes basés grammaire nécessite un énorme travail manuel. En particulier, il est difficile de prendre en compte le contexte en définissant les règles de choix entre les variantes possibles du texte résultat à partir des entrées\cite{Gatt2018}.

\paragraph{Approches statistiques:} Il existe plusieurs méthodes basées sur des statistiques pour la tâche de réalisation. Certains se basent sur des grammaires probabilistes, cette dernière a l’avantage de minimiser le travail manuel tout en couvrant plus de cas de réalisation. Il existe principalement deux approches l’utilisant\cite{Gatt2018}:
\begin{itemize}
	\item La première se base sur une petite grammaire qui génère plusieurs alternatives qui sont ensuite ordonnés selon un modèle statistique basé sur un corpus pour sélectionner la phrase la plus probable (par exemple Langkilde-Geary (2000)\cite{LangkildeGeary2000}).
	\item La deuxième méthode utilise les informations statistiques directement au niveau de la génération pour produire la solution optimale (exemple: Belz (2008)\cite{Belz2008}).
\end{itemize}
	Dans les deux méthodes sus-citées la grammaire de base peut être manuellement faite, dans ce cas, les informations statistiques aideront à la détermination de la solution optimale, ou elle peut être extraite à partir des données, comme l’utilisation des Treebanks \footnote{un Treebank est un texte analysé qui contient des informations syntaxiques ou sémantiques sur les structures de phrases} pour déduire les règles de grammaire\cite{Espinosa2008}.\newline
\paragraph{}
D’autres approches statistiques n’utilisent pas des grammaires mais se basent sur des classificateurs. Ces derniers peuvent être cascadés de telle sorte à décider quel constituant utiliser dans quelle position ainsi que les modifications nécessaires pour générer un texte correcte. À noter qu’une telle approche, ne nécessitant pas l’utilisation de grammaire, utilise des entrées plus abstraites et moins détaillées linguistiquement. À voire même la possibilité de s’étendre aux autres tâches de NLG, c’est à dire un système qui accomplit plusieurs tâches de NLG en parallèle en utilisant les entrées initiales. Dans la suite de ce travail nous allons présenter certains de ces systèmes qui sont plus utilisés récemment.

\subsection{Systèmes basés encodeur-décodeur}
\paragraph{}
Une architecture souvent utilisée dans le traitement du langage naturel est l’encodeur-décodeur (part2). En particulier, son utilisation dans les tâches seq2seq (part2) ce qui permet de mettre en correspondance une séquence de taille variable en entrée avec une autre séquence en sortie. Les modèles seq2seq peuvent être adapter pour convertir une représentation abstraite de l’information en langage naturel\cite{Ferreira2017}.\newline
\begin{figure}[H]
	\centering
	\includegraphics[width=.95\linewidth]{images/NLG/Encoder.png} 
	\caption{Schéma d'une architecture encodeur-décodeur pour NLG} 
\end{figure}
\paragraph{}
Beaucoup d’approches de génération de langage naturel en gestion de dialogue utilise des encodeur-décodeur. Wen et al. (2015)\cite{Wen2015} utilise par exemple des LSTMs(part2) sémantiquement conditionnés; il ajoute aux LSTMs classiques une couche contenant des informations sur l’action prise par le gestionnaire de dialogue pour assurer que la génération représente le sens désiré. D’autres travaux utilisent des réseaux de neurones récurrent (part2) pour encoder l’état du gestionnaire de dialogue et l’entrée reçu suivis par un décodeur pour générer le texte de la réponse\cite{Sordoni2015}\cite{Serban2016}\cite{Goyal2016}.
	